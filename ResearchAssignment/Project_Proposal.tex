
%% Beginning of file 'sample63.tex'
%%
%% Modified 2019 June
%%
%% This is a sample manuscript marked up using the
%% AASTeX v6.3 LaTeX 2e macros.
%%
%% AASTeX is now based on Alexey Vikhlinin's emulateapj.cls 
%% (Copyright 2000-2015).  See the classfile for details.

%% AASTeX requires revtex4-1.cls (http://publish.aps.org/revtex4/) and
%% other external packages (latexsym, graphicx, amssymb, longtable, and epsf).
%% All of these external packages should already be present in the modern TeX 
%% distributions.  If not they can also be obtained at www.ctan.org.

%% The first piece of markup in an AASTeX v6.x document is the \documentclass
%% command. LaTeX will ignore any data that comes before this command. The 
%% documentclass can take an optional argument to modify the output style.
%% The command below calls the preprint style which will produce a tightly 
%% typeset, one-column, single-spaced document.  It is the default and thus
%% does not need to be explicitly stated.
%%
%%
%% using aastex version 6.3
\documentclass[twocolumn]{aastex63}

%% The default is a single spaced, 10 point font, single spaced article.
%% There are 5 other style options available via an optional argument. They
%% can be invoked like this:
%%
%% \documentclass[arguments]{aastex63}
%% 
%% where the layout options are:
%%
%%  twocolumn   : two text columns, 10 point font, single spaced article.
%%                This is the most compact and represent the final published
%%                derived PDF copy of the accepted manuscript from the publisher
%%  manuscript  : one text column, 12 point font, double spaced article.
%%  preprint    : one text column, 12 point font, single spaced article.  
%%  preprint2   : two text columns, 12 point font, single spaced article.
%%  modern      : a stylish, single text column, 12 point font, article with
%% 		  wider left and right margins. This uses the Daniel
%% 		  Foreman-Mackey and David Hogg design.
%%  RNAAS       : Preferred style for Research Notes which are by design 
%%                lacking an abstract and brief. DO NOT use \begin{abstract}
%%                and \end{abstract} with this style.
%%
%% Note that you can submit to the AAS Journals in any of these 6 styles.
%%
%% There are other optional arguments one can invoke to allow other stylistic
%% actions. The available options are:
%%
%%   astrosymb    : Loads Astrosymb font and define \astrocommands. 
%%   tighten      : Makes baselineskip slightly smaller, only works with 
%%                  the twocolumn substyle.
%%   times        : uses times font instead of the default
%%   linenumbers  : turn on lineno package.
%%   trackchanges : required to see the revision mark up and print its output
%%   longauthor   : Do not use the more compressed footnote style (default) for 
%%                  the author/collaboration/affiliations. Instead print all
%%                  affiliation information after each name. Creates a much 
%%                  longer author list but may be desirable for short 
%%                  author papers.
%% twocolappendix : make 2 column appendix.
%%   anonymous    : Do not show the authors, affiliations and acknowledgments 
%%                  for dual anonymous review.
%%
%% these can be used in any combination, e.g.
%%
%% \documentclass[twocolumn,linenumbers,trackchanges]{aastex63}
%%
%% AASTeX v6.* now includes \hyperref support. While we have built in specific
%% defaults into the classfile you can manually override them with the
%% \hypersetup command. For example,
%%
%% \hypersetup{linkcolor=red,citecolor=green,filecolor=cyan,urlcolor=magenta}
%%
%% will change the color of the internal links to red, the links to the
%% bibliography to green, the file links to cyan, and the external links to
%% magenta. Additional information on \hyperref options can be found here:
%% https://www.tug.org/applications/hyperref/manual.html#x1-40003
%%
%% Note that in v6.3 "bookmarks" has been changed to "true" in hyperref
%% to improve the accessibility of the compiled pdf file.
%%
%% If you want to create your own macros, you can do so
%% using \newcommand. Your macros should appear before
%% the \begin{document} command.
%%
\newcommand{\vdag}{(v)^\dagger}
\newcommand\aastex{AAS\TeX}
\newcommand\latex{La\TeX}

%% Reintroduced the \received and \accepted commands from AASTeX v5.2
\received{March 24, 2020}

%% Command to document which AAS Journal the manuscript was submitted to.
%% Adds "Submitted to " the argument.


%% For manuscript that include authors in collaborations, AASTeX v6.3
%% builds on the \collaboration command to allow greater freedom to 
%% keep the traditional author+affiliation information but only show
%% subsets. The \collaboration command now must appear AFTER the group
%% of authors in the collaboration and it takes TWO arguments. The last
%% is still the collaboration identifier. The text given in this
%% argument is what will be shown in the manuscript. The first argument
%% is the number of author above the \collaboration command to show with
%% the collaboration text. If there are authors that are not part of any
%% collaboration the \nocollaboration command is used. This command takes
%% one argument which is also the number of authors above to show. A
%% dashed line is shown to indicate no collaboration. This example manuscript
%% shows how these commands work to display specific set of authors 
%% on the front page.
%%
%% For manuscript without any need to use \collaboration the 
%% \AuthorCollaborationLimit command from v6.2 can still be used to 
%% show a subset of authors.
%
%\AuthorCollaborationLimit=2
%
%% will only show Schwarz & Muench on the front page of the manuscript
%% (assuming the \collaboration and \nocollaboration commands are
%% commented out).
%%
%% Note that all of the author will be shown in the published article.
%% This feature is meant to be used prior to acceptance to make the
%% front end of a long author article more manageable. Please do not use
%% this functionality for manuscripts with less than 20 authors. Conversely,
%% please do use this when the number of authors exceeds 40.
%%
%% Use \allauthors at the manuscript end to show the full author list.
%% This command should only be used with \AuthorCollaborationLimit is used.

%% The following command can be used to set the latex table counters.  It
%% is needed in this document because it uses a mix of latex tabular and
%% AASTeX deluxetables.  In general it should not be needed.
%\setcounter{table}{1}

%%%%%%%%%%%%%%%%%%%%%%%%%%%%%%%%%%%%%%%%%%%%%%%%%%%%%%%%%%%%%%%%%%%%%%%%%%%%%%%%
%%
%% The following section outlines numerous optional output that
%% can be displayed in the front matter or as running meta-data.
%%
%% If you wish, you may supply running head information, although
%% this information may be modified by the editorial offices.

%%
%% You can add a light gray and diagonal water-mark to the first page 
%% with this command:
%% \watermark{text}
%% where "text", e.g. DRAFT, is the text to appear.  If the text is 
%% long you can control the water-mark size with:
%% \setwatermarkfontsize{dimension}
%% where dimension is any recognized LaTeX dimension, e.g. pt, in, etc.
%%
%%%%%%%%%%%%%%%%%%%%%%%%%%%%%%%%%%%%%%%%%%%%%%%%%%%%%%%%%%%%%%%%%%%%%%%%%%%%%%%%
\graphicspath{{./}{figures/}}
%% This is the end of the preamble.  Indicate the beginning of the
%% manuscript itself with \begin{document}.

\begin{document}

\title{The kinematics of the stellar disk particles in the MW/M31 galaxy major merger remnant}

%% LaTeX will automatically break titles if they run longer than
%% one line. However, you may use \\ to force a line break if
%% you desire. In v6.3 you can include a footnote in the title.

%% A significant change from earlier AASTEX versions is in the structure for 
%% calling author and affiliations. The change was necessary to implement 
%% auto-indexing of affiliations which prior was a manual process that could 
%% easily be tedious in large author manuscripts.
%%
%% The \author command is the same as before except it now takes an optional
%% argument which is the 16 digit ORCID. The syntax is:
%% \author[xxxx-xxxx-xxxx-xxxx]{Author Name}
%%
%% This will hyperlink the author name to the author's ORCID page. Note that
%% during compilation, LaTeX will do some limited checking of the format of
%% the ID to make sure it is valid. If the "orcid-ID.png" image file is 
%% present or in the LaTeX pathway, the OrcID icon will appear next to
%% the authors name.
%%
%% Use \affiliation for affiliation information. The old \affil is now aliased
%% to \affiliation. AASTeX v6.3 will automatically index these in the header.
%% When a duplicate is found its index will be the same as its previous entry.
%%
%% Note that \altaffilmark and \altaffiltext have been removed and thus 
%% can not be used to document secondary affiliations. If they are used latex
%% will issue a specific error message and quit. Please use multiple 
%% \affiliation calls for to document more than one affiliation.
%%
%% The new \altaffiliation can be used to indicate some secondary information
%% such as fellowships. This command produces a non-numeric footnote that is
%% set away from the numeric \affiliation footnotes.  NOTE that if an
%% \altaffiliation command is used it must come BEFORE the \affiliation call,
%% right after the \author command, in order to place the footnotes in
%% the proper location.
%%
%% Use \email to set provide email addresses. Each \email will appear on its
%% own line so you can put multiple email address in one \email call. A new
%% \correspondingauthor command is available in V6.3 to identify the
%% corresponding author of the manuscript. It is the author's responsibility
%% to make sure this name is also in the author list.
%%
%% While authors can be grouped inside the same \author and \affiliation
%% commands it is better to have a single author for each. This allows for
%% one to exploit all the new benefits and should make book-keeping easier.
%%
%% If done correctly the peer review system will be able to
%% automatically put the author and affiliation information from the manuscript
%% and save the corresponding author the trouble of entering it by hand.




\author{Madison Walder}



%% Note that the \and command from previous versions of AASTeX is now
%% depreciated in this version as it is no longer necessary. AASTeX 
%% automatically takes care of all commas and "and"s between authors names.

%% AASTeX 6.3 has the new \collaboration and \nocollaboration commands to
%% provide the collaboration status of a group of authors. These commands 
%% can be used either before or after the list of corresponding authors. The
%% argument for \collaboration is the collaboration identifier. Authors are
%% encouraged to surround collaboration identifiers with ()s. The 
%% \nocollaboration command takes no argument and exists to indicate that
%% the nearby authors are not part of surrounding collaborations.

%% Mark off the abstract in the ``abstract'' environment. 

%% From the front matter, we move on to the body of the paper.
%% Sections are demarcated by \section and \subsection, respectively.
%% Observe the use of the LaTeX \label
%% command after the \subsection to give a symbolic KEY to the
%% subsection for cross-referencing in a \ref command.
%% You can use LaTeX's \ref and \label commands to keep track of
%% cross-references to sections, equations, tables, and figures.
%% That way, if you change the order of any elements, LaTeX will
%% automatically renumber them.
%%
%% We recommend that authors also use the natbib \citep
%% and \citet commands to identify citations.  The citations are
%% tied to the reference list via symbolic KEYs. The KEY corresponds
%% to the KEY in the \bibitem in the reference list below. 

\section{Introduction} \label{sec:intro}

As is well known throughout the astronomy community, The Milky Way (MW) and Andromeda (M31) are set on a course to collide in 4 billion years.  Luckily, we can determine what this collision will look like and how the aftermath will behave dynamically through simulations.  This project will be focusing on using simulation data to determine the kinematics of disk stars in the remnant of the MW and M31 major merger event.  I plan to address how the motions of the disk stars from both galaxies contribute to the motion of the remnant as a whole. For example, I plan to determine whether or not the remnant is rotating.  The stellar kinematics to be analyzed will be the velocity of particles, the dispersion of velocities, and its angular momentum as function of radius from the center of the remnant.  

Analyzing the stellar kinematics of a major merger remnant is incredibly important to galaxy evolution because it can help us understand why the kinematics of a certain type of galaxy look the way they do in the present.  In terms of galaxy evolution, studying the motions of a merger remnant allows us to take into account that galaxies can interact with each other which certainly affects how they evolve as opposed to if they evolved without any interactions. They also allow us to visualize how galaxies with similar mass that are on a collision course with each other will behave in the future.   

The theory that motivates projects like this is referred to as "merger hypothesis", which states that the merging of two equal-mass, gas rich spiral galaxies forms an elliptical galaxy \citep{1972ApJ...178..623T}.  This is supported by \cite{2006ApJ...650..791C} who used numerical simulations to study the kinematics of major merger events between gas-rich and gas-poor mergers (referred to as dissipational and dissipationless respectively in the paper).  As shown in Figure 1, they found that the simulations of gas-rich remnants successfully replicated the observed kinematic properties elliptical galaxies, while gas-poor remnants did not. 

\begin{figure}
    \centering
    \includegraphics[scale = 0.55]{Cox_figure.png}
    \caption{Histogram of (V/$\sigma$) for both gas-poor (blue) and gas-rich (red) remnant simulations overplotted with data from observed ellipticals and spheroids \citep{2006ApJ...650..791C}.}
    \label{fig:my_label}
\end{figure}

Many of the open questions in this field have to do with the accuracy of the "merger hypothesis" predicting the correct formation of ellipticals.  For example, one thing it does not account for is the apparent lack of a dark matter halo observed in ellipticals \citep{2003Sci...301.1696R}.  \textit{Where does the dark matter go after the spirals merge?}. For this project, some major open questions I hope to help answer are: \textit{What will the Milky Way and M31 remnant look like?} and \textit{How will the merger remnant behave kinematically?}     

\section{Proposal} \label{sec:style}
\subsection{What specific question(s) will you be addressing?}
I will be focusing on some specific aspects of the stellar kinematics of the MW/M31 merger remnant.  I plan to address the following: whether the merged remnant is rotating, and whether it is a fast or slow rotator if rotation is present.  I will also be comparing the contribution of the Milky Way vs. M31 to the kinematics seen in the remnant, determining its velocity dispersion as a function of radius, and making an angular momentum comparison between the stellar component and the dark matter component of the remnant.       
\subsection{How will you approach the problem using the simulation data?}
The simulation data to be used was generated by \cite{2012ApJ...753....8V}. 

To find whether the merged remnant is rotating, I will create a phase diagram of velocity vs. radius for a snapshot of the system (after the two galaxies have fully merged around 6.5 Gyr) by making use of the MassProfile class we created for Homework 5, specifically the CircularVelocity functions.  I will determine whether it is a slow or fast rotator by calculating the (V/$\sigma$) ratio for the remnant through using the CircularVelocity functions and calculating the velocity dispersion using numpy and using the classification that if (V/$\sigma$) ratio is < 0.6, then it is a slow rotator.

To determine the contribution of the MW vs. M31 to the kinematics of the remnant, I will create velocity profiles for the stellar component of each galaxy before they merged (curve for the Milky Way shown in Figure 2) and compare them to the stellar velocity profile of the remnant to see which one is the most similar.  Therefore, I will be able to see which galaxy's kinematics contributed the most.  

To determine the velocity dispersion of the remnant as a function of radius, I will calculate the mean velocity at each radius and determine its spread using numpy, then store the dispersions in an array and plot them as a function of radius using matplotlib.  

To determine the specific angular momentum of the stellar component of the remnant I will calculate the angular momentum of the disk particles as a function of radius using the MassEnclosed and CircularVelocity functions.  I will then do the same thing but for the dark matter halo particles and compare the results to those obtained for the stellar component.  

\subsection{Figure that demonstrates methodology}

See Figure 2 to the right because it apparently does not want to be here.  I will be creating a rotation curve for M31 before the merger as well, then create another rotation curve for the remnant to compare with the MW and M31 curves.  


\subsection{What is your hypothesis of what you will find? Why do you think this will occur?}

I believe that the MW/M31 remnant will be rotating since both the MW and M31 have their own angular momentum, then there is no possible way it will be devoid of angular momentum.  I also believe that both the MW and M31 will contribute about the same amount to the kinematics of the remnant because they have very similar dark matter and disk star masses as well as similar rotation curves.     

For the velocity dispersion of the remnant as a function of radius, I believe that it will decrease at further radii as seen in \cite{2003Sci...301.1696R}.  Following "merger hypothesis", the merging of two gas-rich spiral galaxies should form an elliptical as a remnant.  Even though this remnant will not be the result of a gas-rich major merger, I believe that the observed dearth of the dark matter in ellipticals will still apply because dark matter does not interact with baryonic matter, so the lack of gas should not have much to do with the seeming disappearance of dark matter when the two galaxies merge.

Following the hypothesis that there will be a lack of dark matter in the remnant, I believe that the specific angular momentum of the stellar remnant will be larger than the angular momentum of the dark matter halo remnant.


\vspace{3mm} %3mm vertical space
\begin{figure}[h]
    \centering
    \includegraphics[scale = 0.36]{RotationCurve.png}
    \caption{Rotation curve for the Milky Way before it merges with Andromeda including all components of the galaxy.}
    \label{fig:my_label}
\end{figure}

%% For this sample we use BibTeX plus aasjournals.bst to generate the
%% the bibliography. The sample63.bib file was populated from ADS. To
%% get the citations to show in the compiled file do the following:
%%
%% pdflatex sample63.tex
%% bibtext sample63
%% pdflatex sample63.tex
%% pdflatex sample63.tex

\bibliography{sample63}{}
\bibliographystyle{aasjournal}


%% This command is needed to show the entire author+affiliation list when
%% the collaboration and author truncation commands are used.  It has to
%% go at the end of the manuscript.
%\allauthors

%% Include this line if you are using the \added, \replaced, \deleted
%% commands to see a summary list of all changes at the end of the article.
%\listofchanges

\end{document}

% End of file `sample63.tex'.
